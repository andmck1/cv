\documentclass[letterpaper,11pt]{article}

\usepackage{latexsym}
\usepackage[empty]{fullpage}
\usepackage{titlesec}
\usepackage{marvosym}
\usepackage[usenames,dvipsnames]{color}
\usepackage{verbatim}
\usepackage{enumitem}
\usepackage[hidelinks]{hyperref}
\usepackage{fancyhdr}
\usepackage[english]{babel}
\usepackage{tabularx}
\usepackage{xcolor}
\usepackage{fontawesome5}

\input{glyphtounicode}

% -------------------- FONT OPTIONS --------------------
% sans-serif
% \usepackage[sfdefault]{roboto}
% \usepackage[sfdefault]{noto-sans}
% serif
% \usepackage{charter}

\pagestyle{fancy}
\fancyhf{} % clear all header and footer fields
\fancyfoot{}
\renewcommand{\headrulewidth}{0pt}
\renewcommand{\footrulewidth}{0pt}

% Adjust margins
\addtolength{\oddsidemargin}{-0.5in}
\addtolength{\evensidemargin}{-0.5in}
\addtolength{\textwidth}{1in}
\addtolength{\topmargin}{-.5in}
\addtolength{\textheight}{1.0in}

\urlstyle{same}

\raggedbottom
\raggedright
\setlength{\tabcolsep}{0in}

% Section formatting
\titleformat{\section}{
    \vspace{-5pt}\scshape\raggedright\large
}{}{0em}{}[\color{black}\titlerule \vspace{-5pt}]

% Subsection formatting
\titleformat{\subsection}{
    \vspace{-4pt}\scshape\raggedright\large
}{\hspace{-.15in}}{0em}{}[\color{black}\vspace{-8pt}]

% Ensure that generate pdf is machine readable/ATS parsable
\pdfgentounicode=1

% -------------------- START OF DOCUMENT --------------------
\begin{document}

% -------------------- HEADING--------------------
\vspace{-5pt}

\begin{center}
    \textbf{\Huge \scshape Alastair Hamilton} \\
    \textit{\small +44 0770 775 1450 $|$ alastair.amhamilton@icloud.com $|$ Dual British-Irish Citizenship}
    \vspace{8pt} 
\end{center}

% -------------------- SUMMARY --------------------
\section{Summary}
\begin{itemize}[leftmargin=0.2in, label={}]
    \item 
        \begin{minipage}[t]{1.0\linewidth}
            Having spent the past two years building a data science function
            from scratch at a successful sustainability start-up,
            I'm now looking for a new challenge where I can learn about and enjoy
            the field of ML and AI among a group of like-minded colleagues. 
            Outside of work I enjoy playing my piano, gaming and whatever niche
            hobby I have at the time.
        \end{minipage}
\end{itemize}

% -------------------- EDUCATION --------------------
\section{Education}
\begin{itemize}[leftmargin=0.15in, label={}]
    \item
    \begin{tabular*}{0.97\textwidth}[t]{l@{\extracolsep{\fill}}r}
        \textbf{University of Edinburgh} & June 2017 \\
        \textit{\small BSc Theoretical Physics} & \textit{\small 2:1 (69\%)} \\
    \end{tabular*}
\end{itemize}


% -------------------- WORK EXPERIENCE --------------------
\section{Employment}
\begin{itemize}[leftmargin=0.15in, label={}]
    \item
    \begin{minipage}[t]{0.75\linewidth}
        \textbf{Altruistiq} $|$ Sustainability \\
    \end{minipage} \hfill \textit{\small May 2022 -- Present} \\
    \vspace{-7pt}
    \textit{\small Senior Lead Data Scientist}
    \vspace{-7pt}
    \begin{itemize}[label={\tiny\textbullet}]
        \item
        \begin{minipage}[t]{0.75\linewidth}
            \item\small 
                Semi-automated PDF data extraction using a pipeline 
                of models for clustering, grouping, multi-label multi-class 
                classification and text extraction deployed in AWS Sagemaker.
            \item\small
                Designed the level 4 MLOps architecture including: 
                experimentation and model benchmarking;
                model versioning; model release; data versioning using a 
                feature store; data pipelining; and model monitoring and 
                automatic model retraining. This was done using MLflow,
                AWS Sagemaker, Github Actions and Metabase.
            \item\small
                Developed and led the code development for my team's
                codebase and internal tooling, including all 
                DevOps for library releases across the company and the 
                terraform for any infrastructure required.
            \item\small 
                Developed and deployed a ChatGPT-based Streamlit app
                to enable analysts to navigate complex PDFs in multiple
                languages using a privately hosted Llama-2-13b model for 
                data privacy.
            \end{minipage}
    \end{itemize}

    \item
    \textit{\small Lead Data Scientist}
    \vspace{-7pt}
    \begin{itemize}[label={\tiny\textbullet}]
        \item
            \begin{minipage}[t]{0.75\linewidth}
                \item\small
                    Worked with product managers to break down complex data science
                    and ML project requirements into a clear backlog that could
                    be managed in sprint cycles by non-technical managers
                    and be prioritised against other ongoing company projects.
                \item\small 
                    Developed and deployed a vector-based retrieval algorithm 
                    using BERT embeddings for automating the tagging of 
                    customer data.
                \item\small 
                    Researched representing the current tag hierarcy as a
                    knowledge graph, comparing an RDF and Apache TinkerPop 
                    property graph deployed on AWS Neptune.
            \end{minipage}
    \end{itemize}\vspace{0.2cm}

    \item
    \begin{minipage}[t]{0.75\linewidth}
        \textbf{Lloyds Banking Group} $|$ Financial Services \\
    \end{minipage} \hfill \textit{\small Sep 2018 -- May 2022} \\
    \vspace{-7pt}
    \textit{\small Data Scientist}
    \vspace{-7pt}
    \begin{itemize}[label={\tiny\textbullet}]
        \item
            \begin{minipage}[t]{0.75\linewidth}
                \item\small
                    Developed a novelty detection algorithm in Azure using the CluStream framework as part of the bank's
                    insider criminal threat detection project.
                \item\small
                    Authored and supported a Data Science Masters project at the 
                    University of Edinburgh, focusing on utilising Bayesian Neural Networks 
                    for anomaly detection.
            \end{minipage}
    \end{itemize}

    \item
    \textit{\small Cognitive Data Science Manager}
    \vspace{-7pt}
    \begin{itemize}[label={\tiny\textbullet}]
        \item
            \begin{minipage}[t]{0.75\linewidth}
                \item\small
                    Developed and embedded a topic modelling tool using unsupervised LDA topic models
                    for identifying emerging themes in customer vulnerability documents, resulting in successful 
                    identification of OOS climate change risks.
            \end{minipage}
    \end{itemize}\vspace{0.2cm}
  
    \item
    \begin{minipage}[t]{0.75\linewidth}
        \textbf{BP} $|$ Oil \& Gas \\
    \end{minipage} \hfill \textit{\small Sep 2017 -- Jun 2018} \\
    \vspace{-7pt}
    \textit{\small Supply and Trading Market Intelligence Analyst (Graduate)}
    \vspace{-7pt}\vspace{0.2cm}

\end{itemize}

% -------------------- PUBLICATIONS --------------------

\section{Publications}
\begin{itemize}[leftmargin=0.2in, label={}]
    \item 
        \begin{minipage}[t]{0.75\linewidth}
            Thompson, S., Teixeira-Dias, F., Paulino, M., \& Hamilton, A. \\
            \textit{Predictions on multi-class terminal ballistics datasets using conditional Generative Adversarial Networks.} \\
            Neural Networks, 154, 425-440.
        \end{minipage} \hfill \textit{2022}
\end{itemize}

\begin{itemize}[leftmargin=0.2in, label={}]
  \item
      \begin{minipage}[t]{0.75\linewidth}
          Thompson, S., Teixeira-Dias, F., Paulino, M., \& Hamilton, A. \\
          \textit{Ballistic response of armour plates using Generative Adversarial Networks.} \\
          Defence Technology, 18(9), 1513-1522.
      \end{minipage} \hfill \textit{2022}
\end{itemize}


\end{document}

